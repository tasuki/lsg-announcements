\documentclass[a4paper]{article}

\usepackage[margin=2.5cm]{geometry}
\usepackage{polski}
\usepackage{fontspec}
\usepackage{multicol}
\usepackage{scrextend}
\usepackage{ragged2e}
\usepackage{tabularx}

\setmainfont{Eczar}
\setmonofont{Fira Mono}

\pagestyle{empty}
\parindent 0pt
\changefontsizes[24pt]{18pt}

\newcommand{\maintitle}[1]{{
\bf\fontsize{32pt}{32pt}\selectfont#1\vskip 10pt
}}

\newcommand{\subtitle}[1]{{
\bf\fontsize{24pt}{24pt}\selectfont#1\vskip 5pt
}}

\newcommand{\daytitle}[1]{{
\bf\fontsize{24pt}{24pt}\selectfont\parindent 1cm #1\vskip 5pt
}}

\newcommand{\li}[1]{{
\parindent -1em{--\enspace#1}

}}

\newcolumntype{R}[1]{>{\raggedleft\arraybackslash}p{#1}}
\newcolumntype{L}[1]{>{\raggedright\arraybackslash}p{#1}}

\newcommand{\schedule}[1]{
\begin{tabular}{ R{1.5cm} L{16cm} }
#1
\end{tabular}
}

\newcommand{\prizes}[1]{
\begin{tabular}{ l r }
#1
\end{tabular}
}


\begin{document}

\maintitle{Konkurs tsumego}
\li{Dwie edycje. Rozstrzygnięcie pierwszej w {\it piątek 15.07}, drugiej w {\it czwartek 21.07.}}
\li{Rozwiązania problemów w postaci wypełnionej kartki oddawać Siasiowi.}
\li{{\it Wpisowe} za każdy zgłoszony problem to {\it 15 Łosi.}}
\li{Każdy może zgłosić rozwiązania {\it maksymalnie 3 problemów}.}
\li{Na karteczce należy się podpisać, zaznaczyć pierwszy ruch i ok\-reś\-lić końcowy rezultat.}
\li{{\it Ko ko nierówne}. W przypadku nietypowych ko należy opisać, co to za rodzaj. Np. gdy czarny musi wygrać ko dwukrotnie, żeby prze\-żyć, jest to dwustopniowe ko.}

\vfill

\subtitle{Przebieg rozstrzygnięcia konkursu}
\li{Do każdego problemu wywołani zostają dwaj uczestnicy, którzy zgłosili ten problem, jednak mają inną odpowiedź.}
\li{Wywołani uczestnicy odbywają pojedynek -- każdy próbuje udowodnić słuszność własnego rozwiązania.}
\li{Zwycięzca pojedynku wygrywa wpisowe przegranego.}
\li{Jeśli są uczestnicy, którzy nie zgadzają się z przedstawionym rozwiązaniem, to rzucają wyzwanie przedstawiającemu i od\-by\-wa\-ją taki sam pojedynek.}
\li{Na koniec niepokonanemu w pojedynkach uczestnikowi może rzucić wyzwanie Siasio.}
\li{Wpisowe uczestników, którzy nie zostali pokonani w pojedynku, zostaje dodane do puli nagród.}
\li{Pula nagród zostaje rozdzielona pomiędzy uczestników, którzy zgłosili poprawne rozwiązanie problemu.}

\end{document}
