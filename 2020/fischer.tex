\documentclass[a4paper]{article}

\usepackage[margin=2.5cm]{geometry}
\usepackage{polski}
\usepackage{fontspec}
\usepackage{multicol}
\usepackage{scrextend}
\usepackage{ragged2e}
\usepackage{tabularx}

\setmainfont{Eczar}
\setmonofont{Fira Mono}

\pagestyle{empty}
\parindent 0pt
\changefontsizes[24pt]{18pt}

\newcommand{\maintitle}[1]{{
\bf\fontsize{32pt}{32pt}\selectfont#1\vskip 10pt
}}

\newcommand{\subtitle}[1]{{
\bf\fontsize{24pt}{24pt}\selectfont#1\vskip 5pt
}}

\newcommand{\daytitle}[1]{{
\bf\fontsize{24pt}{24pt}\selectfont\parindent 1cm #1\vskip 5pt
}}

\newcommand{\li}[1]{{
\parindent -1em{--\enspace#1}

}}

\newcolumntype{R}[1]{>{\raggedleft\arraybackslash}p{#1}}
\newcolumntype{L}[1]{>{\raggedright\arraybackslash}p{#1}}

\newcommand{\schedule}[1]{
\begin{tabular}{ R{1.5cm} L{16cm} }
#1
\end{tabular}
}

\newcommand{\prizes}[1]{
\begin{tabular}{ l r }
#1
\end{tabular}
}


\begin{document}

\maintitle{Czas według Bobbiego Fischera}

Każdy z graczy dostaje podstawowy czas i za każdy zagrany ruch dostaje dodatkowy bonus. W turniejach klasy B po 30 minut na początek oraz 10 sekund za każdy wykonany ruch. W turniejach klasy A po 45 minut na początek + 15 sekund za ruch.

\vskip 30pt
\subtitle{Dlaczego?}

Bo japońskie byoyomi jest złe: Gracz nie może oszczędzać czasu. Jeżeli ktoś chce skutecznie wykorzystywać czas, ciągle gra na granicy przegrania przez czas.

\vskip 30pt
\subtitle{Jak?}

Fischera można grać na nowych ,,DGT2010'' zegarach, na starych ,,INGowych'' zegarach niestety nie jest dostępny.

Zegar pamięta poprzednie ustawienia, żeby je powtórzyć, włączamy zegar i cały czas klikamy guzik po prawej ,,OK'' póki nie jest gotowe. Jeżeli gracze przed nami używali innych ustawień, trzeba ustawić ręcznie:

\li{Włączamy zegar guzikiem na dole i ustawiamy program 18.}
\li{Guzikami +/- i OK ustawiamy godziny i minuty podstawowego czasu (dla klasy A jest to 0:45) oraz sekundy podstawowego czasu (generalnie .00) dla pierwszego i potem drugiego gracza.}
\li{Ustawiamy minuty i sekundy dodane za każdy ruch (dla klasy A jest to 0.15) dla pierwszego i potem dla drugiego gracza.}
\li{Jak jest gotowe, możemy włączyć dzwięk guzikiem z nutą po lewej stronie. Zegar uruchamiamy (ew. zatrzymujemy) guzikiem ,,Play'' po środku.}

\end{document}
